\documentclass{article} 
\usepackage[UTF8]{ctex}

\usepackage[utf8]{inputenc}
\usepackage{fontawesome}
\usepackage[a4paper]{geometry}
\usepackage{amsmath}
\usepackage{indentfirst}
\usepackage{graphicx}

\title{如何衡量智能}
\author{wang}
\date{\today}
\begin{document}
\maketitle

论文题目: On the Measure of Intelligence
论文地址: https://arxiv.org/pdf/1911.01547.pdf
论文发表于: arxiv

这篇论文主要探讨了如何衡量智能的问题,然后作者最后提出了ARC数据集用于测量智能的抽象和逻辑推理能力。

\section{如何定义智能?}
论文的第一章主要是对过去研究的回顾和总结,作者认为虽然现在AI在特定任务上取得了很优秀的成绩,但是一直未能取得理想的成绩,并且有着很多缺陷,比如易受对抗样本影响,数据饥渴,对训练样本之外的知识一无所知,需要人类干预才能表现智能,这些问题指向了对于智能的定义和衡量问题。

作者回顾了过去的研究,将对AI的定义归纳为两大观点:

\item AI是特定任务解决方案的集合。由这个观点产生了知识库等。

\item AI代表通用的学习能力

而基于以上两个观点,对于AI的评估方法也有区别和变化

\end{document}
